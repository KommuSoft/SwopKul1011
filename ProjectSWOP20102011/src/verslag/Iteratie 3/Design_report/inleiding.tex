Dit verslag beschrijft de derde iteratie van het dispatch systeem voor het vak \textit{Software-Ontwerp}.
\paragraph{}
%nadruk en nieuwigheden vermelden
Het verslag is als volgt gestructureerd. We beginnen met het vernieuwde domeinmode in sectie \ref{domeinmodel}. De terminologie van dit domeinmode wordt verder uitgelegd in sectie \ref{terminologie}. In sectie \ref{overzicht} geven we een overzicht van het systeem en zijn subsystemen.
We tonen ook hoe we de verschillende verantwoordelijkheden hebben toegewezen.
In sectie \ref{patterns} bespreken we de invloed van \textit{GRASP} en \textit{Design Patterns} op ons ontwerp.
Alternatieve ontwerpen die niet gekozen werden, zullen ook vermeld worden.
De mogelijkheden die voorzien zijn voor uitbreidingen zullen besproken worden in sectie \ref{uitbreidbaarheid}.
Vervolgens tonen we in sectie \ref{testen} hoe we onze testen hebben aangepakt.
In sectie \ref{projectbeheer} vermelden we hoe we de verschillende taken hebben verdeeld onder de groepsleden
en hoeveel tijd er werd besteed aan deze taken.
Ten slotte besluiten we in sectie \ref{besluit} door o.a. enkele problemen te schetsen die we zijn tegengekomen.
