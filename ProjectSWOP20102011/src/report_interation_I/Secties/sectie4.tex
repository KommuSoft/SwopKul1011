\section{Uitbreidbaarheid}
\label{uitbreidbaarheid}
Het project is gebouwd vanuit het idee dat het project enkele standaard aanpassingen moet kunnen verwerken, zonder veel code te wijzigen. Een voorbeeld hiervan is de implementatie van de emergencies, waarbij men zelfs de creatie van de Emergency op een generische manier in de user interface tracht te behandelen. Een oorzaak van deze uitbreidbaarheid is Low Coupling. Immers houdt Low Coupling in dat veranderingen op bepaalde plaatsen maar zeer beperkt effect hebben op de code op andere plaatsen. Uitbreidbaarheid streeft dus in sommige gevallen naar hetzelfde als de GRASP Patterns.
\paragraph{}
Welke uitbreidingen hebben we dan in gedachten? In de eerste plaats doet het model ons vermoeden dat de types Emergencies wel eens uitgebreid kunnen worden. De lijst van emergencies was immers beperkt. In het geval we een nieuw type emergency willen toevoegen. Indien we een vulkaanuitbarsting toevoegen, hoeven we immers alleen een subklasse \verb+VolcanicEruption+ te implementeren, en een bijbehorende factory, \verb+VolcanicEruptionFactory+ toe te voegen aan onze lijst van Emergency factories. Indien deze vulkaanuitbarsting geen extra types argumenten bevat die nog niet gekend zijn, hoeft verder helemaal geen code geschreven te worden. Indien we wel nieuwe types parameters toevoegen, zoals het type vulkaan, hoeven we alleen een klasse \verb+VolcanoType+ in onze domeinlaag te implementeren, en een \verb+VolcanoTypeParser+ te schrijven in onze userinterfacelaag. Dergelijk model leidt tot zeer lage koppeling. Immers dient geen andere klasse in de domeinlaag van het bestaan van een vulkaanuitbarsting af te weten. Er wordt dus abstractie gemaakt van het type Emergency.
\subparagraph{}
Een analoge uitbreiding is deze van de \verb+UnitBuilding+ types. Opnieuw kunnen we hier met een gelijkaardig scenario eenvoudig een extra \verb+Unit+ of \verb+Building+ toe te voegen.
\paragraph{}
Ook de userinterface kan verder uitgebreid worden. Hierbij wordt door middel van polymorfisme een framework geleverd, waarbij men makkelijk nieuwe commando's kan toevoegen. Daar we er vanuit gaan dat het aantal use cases nog zal toenemen, is het aanbieden van dergelijke klassen dus te rechtvaardigen.
\paragraph{}
Heel wat code in dit project is overbodig, en levert geen bijdrage. Meestal echter heeft dit tot doel om in geval van nieuwe specificaties makkelijk de nieuwe feiten te implementeren. Meestal wordt dit verwezenlijkt door het invoeren van een abstract niveau alvorens de uiteindelijk implementatie te schrijven.
\paragraph{}
Een concreet voorbeeld is bijvoorbeeld de \verb+EmergencyValidationCriterium+ klasse. Waarbij een abstractie niveau gemaakt wordt van het selecteren van emergencies op status. Deze aanpassing heeft echter ook andere redenen, bijvoorbeeld omwille van het Information Expert GRASP pattern. Een andere maatregel is het invoeren van de \verb+EmergencyFactory+ klasse, om het generisch (en zelfs reflexief) kunnen toevoegen van types emergencies.
\paragraph{}
Verder stelden we vast dat Low Coupeling in de GRASP Patterns meestal uitbreidbaarheid impliceert. Immers zal bij een Low Coupeling van verschillende details abstractie gemaakt kunnen worden. Deze abstracties kunnen dan later concreet ingevuld worden nieuwe klassen.