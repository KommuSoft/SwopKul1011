\label{overzicht}
\subsection{Algemeen}
We deelden het project op in 13 verschillende packages:
\begin{itemize}
	\item de default package (\texttt{projectswop20102011})
  \item de controller package (\texttt{projectswop20102011.controllers})
  \item de domein package (\texttt{projectswop20102011.domain})
  \item de domein list package (\texttt{projectswop20102011.domain.list})
  \item de domein validators package (\texttt{projectswop20102011.domain.validators})
  \item de event handler package (\texttt{projectswop20102011.eventhandlers})
  \item de exception package (\texttt{projectswop20102011.exceptions})
  \item de external system package (\texttt{projectswop20102011.externalsystem})
  \item de external system adapter package (\texttt{projectswop20102011.externalsystem.adapters})
  \item de factories package (\texttt{projectswop20102011.factories}
  \item de userinterface package (\texttt{projectswop20102011.userinterface})
  \item de utils package (\texttt{projectswop20102011.utils})
  \item de utils parsers package (\texttt{projectswop20102011.utils.parsers})
\end{itemize}
Deze opdeling gebeurde om de verschillende lagen van elkaar te onderscheiden. Zo kunnen we bepaalde zichtbaarheidsrechten afdwingen. We zullen nu de verschillende lagen meer in detail bespreken. In de bijlage bevinden zich de system sequence diagrams en de sequence diagrams.

\subsection{Domeinlaag}
 Het klassendiagram van de domeinlaag bevindt zich op figuur \ref{fig:DomeinlaagBeknopt} op pagina \pageref{fig:DomeinlaagBeknopt}. Een uitgebreider UML diagram bevindt zich in de bijlage. Voor de eenvoud is het opgesplitst in drie verschillende delen. Centraal hierin bevindt zich de klasse \texttt{Sendable}. Deze klasse bevat twee subklassen, namelijk \texttt{Emergency} en \texttt{Disaster}. Onder de klasse \texttt{Emergency} komen dan de verschillende types emergencies. Namelijk de klassen \texttt{Fire}, \texttt{PublicDisturbance}, \texttt{Robbery} en \texttt{TrafficAccident}. Ze vormen elk een subklasse met hun specifieke attributen. De klasse \texttt{Disaster} representeert een ramp. Een andere belangrijke klasse is de \texttt{Unit} klasse. Deze bevat de specifieke data over een eenheid. Omdat een \texttt{Unit} en gebouwen zoals een \texttt{Hospital} enkele eigenschappen gemeen hebben (zoals een naam en een locatie), bestaat er een superklasse die ze generaliseert. Namelijk de klasse \texttt{MapItem}. Een klasse \texttt{Building} werd overwogen, maar levert hier nog geen toegevoegde waarde. De klasse \texttt{Unit} wordt verder gespecifieerd in verschillende typen eenheden, namelijk een \texttt{Firetruck}, \texttt{Ambulance} of \texttt{Policecar}. Het is uiteraard de bedoeling dat deze units toegewezen kunnen worden aan een bepaalde \texttt{Sendable}. Deze verantwoordelijkheid wordt vervuld door de klasse \texttt{UnitsNeeded}. De klasse heeft twee kinderen: \texttt{ConcreteUnitsNeeded} \& \texttt{DerivedUnitsNeeded}. \texttt{ConcreteUnitsNeeded} houdt bij welke \texttt{Unit}s  er aan het werken zijn voor een specifieke \texttt{Emergency} en welke al klaar zijn. Verder bevat deze klasse ook een \texttt{Constraint} en een \texttt{Policy}. Deze zorgen voor een juiste assignatie van de \texttt{Unit}s. De klasse \texttt{DerivedUnitsNeeded} weet welke \texttt{Disaster} behandeld wordt en welke constraints er op van toepassing zijn. Ook bestaat er een interface \texttt{TimeSensitive}. Deze duidt aan dat een klasse die deze interface implementeert, afhankelijk is van de tijd. Een ander belangrijk aspect aan de domeinlaag is dat er geen klassen zoals \texttt{Caller} of \texttt{Call} of \texttt{Operator} bestaan. Deze bestaan alleen in het domeinmodel. Maar aangezien er nergens een use case bestaat die hiervan gebruik maakt, zijn ze overbodig.
 
\subsection{Domein listlaag}
Deze laag bevat zeven klassen namelijk \texttt{Disaster\-List}, \texttt{Emergency\-Factory\-List}, \texttt{Emergency\-List}, \texttt{Generic\-Factory\-List}, \texttt{Map\-Item\-List}, \texttt{Parser\-List} \texttt{Time\-Sensitive\-List}. De \texttt{World} bevat deze klassen.

\subsection{Domein validatorslaag}
In deze laag hebben we een aantal klassen die checks uitvoeren. Het eerste onderdeel dat we zullen bespreken is de \texttt{DispatchUnitsConstraint}. Dit is de basisklasse van het \textit{composite} pattern dat we gebruiken. De klasse \texttt{UnitToEmergencyDistanceComparator} vergelijkt de afstand die twee units moeten afleggen om tot hun emergency te geraken. De \texttt{UnitToDisasterDistanceComparator} doet iets gelijkaardigs voor een \texttt{Disaster}. De interface \texttt{MapItemEvaluationCriterium} specifieert een criterium om te controleren of \texttt{MapItem}s valide zijn.

\subsection{External systemlaag}
De external systemlaag legt de link tussen ons programma en een specifieke interface. Deze laag bevat \'e\'en klasse namelijk \texttt{EmergencyDispatchApi}.

\subsection{External system adapterlaag}
Deze package bevat de adapters voor het external system. Deze zorgen ervoor dat ons domein niet verstrengeld raakt met een extern systeem.

\subsection{Factorieslaag}
Om het model makkelijk uitbreidbaar te maken, waarbij zelfs bijvoorbeeld geen rekening met de user interface moet worden gehouden, hebben we een \texttt{EmergencyFactory} ge\"implementeerd. Vervolgens kan onze api een factory aanmaken en met behulp van een array van objecten de \texttt{Emergency} aanmaken. Analoog werden ook factories voor MapItems voorzien. Deze heeft nog twee subklassen, \'e\'en abstracte namelijk de \texttt{UnitFactory}, de andere is dan de \texttt{HopitalFactory}.

\subsection{Controllerlaag}
Vervolgens bevat de controller package alle controllers die de domeinlaag aansturen. Hierbij beschouwen we een abstracte klasse \texttt{Controller}. Waardoor alle controllers een link naar het \texttt{World} object bevatten. Deze wereld kunnen ze dan vervolgens manipuleren. De controllers zijn behoorlijk gelijklopend met de use cases. Voor de initialisatie van de omgeving werd een aparte controller \texttt{ReadEnvironmentDataController} geschreven. Deze is in staat om \texttt{Unit}s en \texttt{Hospital}s aan de wereld toe te voegen.

\subsection{Exceptions, Main, Userinterface en Utils}
Om de code overzichtelijk te houden, werd geopteerd voor een aparte package \texttt{projectswop20102011.\-exceptions}. Deze omvat alle exceptions in het volledige project. 
\todo{Is dit waar?}
Aangezien deze dus op alle lagen kunnen voorkomen, blijft dit de afscherming van de domeinlaag garanderen. Vervolgens bevat de userinterface laag verschillende klassen die de communicatie met de gebruiker mogelijk maken. Deze klassen staan in een gelaagde structuur, om makkelijk nieuwe use cases te implementeren. De \texttt{Main} klasse staat in de default package. Ten slotte hebben we nog een utilslaag die hulpmethoden voorziet.
